
((digest . "ca755e2725f541d76c7d7b33ba0fbe81") (undo-list nil ("t" . -725) ((marker . 725) . -1) (725 . 726) (t 22786 1213 0 0) nil (591 . 595) (t 22786 561 0 0) nil (1167 . 1175) nil (1 . 1255) nil ("
" . 1) nil ("\\end{description}" . 1) nil ("
" . 1) nil ("\\item " . 1) nil ("
" . 1) nil ("\\begin{description}" . 1) nil (1 . 46) nil ("\\documentclass[9pt,fleqn]{jarticle} %記事や論文を書く際にjarticleを使用

%枠付文章とpng画像を出力するため
\\usepackage{ascmac}%枠付き文字出力
\\usepackage{fancybx}
\\usepackage{framed}
\\usepackage[dvipdfmx]{graphicx}%pdf出力のための
\\usepackage{listings,jlisting}
\\usepackage{amsmath,amssymb}
\\usepackage{mediabb}
\\usepackage{titlesec}
\\titleformat*{\\section}{\\normalsize\\bfseries}
\\titleformat*{\\subsection}{\\normalsize\\bfseries}
\\usepackage{url}

\\begin{document}
\\begin{titlepage}
    \\begin{center} %vspaceは縦方向のスペースで,hspaceは横方向のスペース
        \\fontsize{25pt}{0pt}\\selectfont %フォントサイズと行送りを設定
        \\vspace*{100truept}
        \\bf{}\\\\ %タイトル1行目
        \\vspace*{10truept}
        \\bf{レポート\\#2}\\\\ 
        \\vspace{240truept}
    \\end{center}
    \\begin{flushright}
        {\\large
            \\fontsize{16pt}{0pt}\\selectfont
          所  属  :琉球大学 工学部 情報工学科\\\\
            \\vspace{5truept}
          学籍番号:155755G\\\\
            \\vspace{5truept}
          氏  名  :佐藤 匠\\\\
            \\vspace{5truept}
          提 出 日:\\\\
            \\vspace{5truept}
        }    
    \\end{flushright}
\\end{titlepage}

\\tableofcontents{\\tiny} %目次を自動作成するコマンド
\\setcounter{page}{1} %ページ数を1とする
\\pagestyle{plain}

\\newpage

\\section{asd}





\\begin{thebibliography}{9} 
  \\bibitem{} \\url{}
\\end{thebibliography}
\\end{document}% Created 2017-04-15 土 17:50
\\documentclass[11pt]{article}
\\usepackage[utf8]{inputenc}
\\usepackage[T1]{fontenc}
\\usepackage{fixltx2e}
\\usepackage{graphicx}
\\usepackage{longtable}
\\usepackage{float}
\\usepackage{wrapfig}
\\usepackage{rotating}
\\usepackage[normalem]{ulem}
\\usepackage{amsmath}
\\usepackage{textcomp}
\\usepackage{marvosym}
\\usepackage{wasysym}
\\usepackage{amssymb}
\\usepackage{hyperref}
\\tolerance=1000
\\author{Takumi Sato}
\\date{\\today}
\\title{note}
\\hypersetup{
  pdfkeywords={},
  pdfsubject={},
  pdfcreator={Emacs 25.1.1 (Org mode 8.2.10)}}
\\begin{document}

\\maketitle
\\tableofcontents


 pythonでの機械学習ががバズり初めたときに,なにかのパッケージのチュートリアルで
pythonで為替レートの予想をする,というものがあった.
 これは,かなり興味をそそられたが,結局「ゼロから分かるDeeplearning」をやって理解した気になって,終わってしまった
 これではいけないと思い,どうしようかと考えていると,rebuild.fmでpythonは今,データ分析のワークフローが既に出来上がってしまっている,というのを聞いた.なんだかこれなら行けそうという軽い気持ちでやってみようと思う

 データが与えられて,アルゴリズムの詳細の理解とまでは行かずとも,なんとなく応用できる.データの整形から,モデル化,解決案の提案,
といった現実社会であるであろうことをできるようになる.
% Emacs 25.1.1 (Org mode 8.2.10)
\\end{document}
" . 1) ((marker . 59) . -2255) ((marker . 1) . -1856) ((marker . 1) . -1856) ((marker . 1) . -1856) ((marker . 1) . -2255) ((marker . 1) . -712) ((marker) . -2255) nil (1 . 1255) (t 22769 57017 0 0) nil undo-tree-canary))
