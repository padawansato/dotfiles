% Created 2017-04-15 土 17:50
\documentclass[11pt]{article}
\usepackage[utf8]{inputenc}
\usepackage[T1]{fontenc}
\usepackage{fixltx2e}
\usepackage{graphicx}
\usepackage{longtable}
\usepackage{float}
\usepackage{wrapfig}
\usepackage{rotating}
\usepackage[normalem]{ulem}
\usepackage{amsmath}
\usepackage{textcomp}
\usepackage{marvosym}
\usepackage{wasysym}
\usepackage{amssymb}
\usepackage{hyperref}
\tolerance=1000
\author{Takumi Sato}
\date{\today}
\title{note}
\hypersetup{
  pdfkeywords={},
  pdfsubject={},
  pdfcreator={Emacs 25.1.1 (Org mode 8.2.10)}}
\begin{document}

\maketitle
\tableofcontents


 pythonでの機械学習ががバズり初めたときに,なにかのパッケージのチュートリアルで
pythonで為替レートの予想をする,というものがあった.
 これは,かなり興味をそそられたが,結局「ゼロから分かるDeeplearning」をやって理解した気になって,終わってしまった
 これではいけないと思い,どうしようかと考えていると,rebuild.fmでpythonは今,データ分析のワークフローが既に出来上がってしまっている,というのを聞いた.なんだかこれなら行けそうという軽い気持ちでやってみようと思う

 データが与えられて,アルゴリズムの詳細の理解とまでは行かずとも,なんとなく応用できる.データの整形から,モデル化,解決案の提案,
といった現実社会であるであろうことをできるようになる.
% Emacs 25.1.1 (Org mode 8.2.10)
\end{document}
