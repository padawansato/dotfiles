\documentclass[9pt,fleqn]{jarticle} %記事や論文を書く際にjarticleを使用

%枠付文章とpng画像を出力するため
\usepackage{ascmac}%枠付き文字出力
\usepackage{fancybx}
\usepackage{framed}
\usepackage[dvipdfmx]{graphicx}%pdf出力のための
\usepackage{listings,jlisting}
\usepackage{amsmath,amssymb}
\usepackage{mediabb}
\usepackage{titlesec}
\titleformat*{\section}{\normalsize\bfseries}
\titleformat*{\subsection}{\normalsize\bfseries}
\usepackage{url}

\begin{document}
\begin{titlepage}
    \begin{center} %vspaceは縦方向のスペースで,hspaceは横方向のスペース
        \fontsize{25pt}{0pt}\selectfont %フォントサイズと行送りを設定
        \vspace*{100truept}
        \bf{ソフトコンピューティング}\\ %タイトル1行目
        \vspace*{10truept}
        \bf{4.24 TSPのコーディングを考える}\\ 
        \vspace{240truept}
    \end{center}
    \begin{flushright}
        {\large
            \fontsize{16pt}{0pt}\selectfont
          所  属  :琉球大学 工学部 情報工学科\\
            \vspace{5truept}
          学籍番号:155755G\\
            \vspace{5truept}
          氏  名  :佐藤 匠\\
            \vspace{5truept}
          提 出 日:2017-05-01\\
            \vspace{5truept}
        }    
    \end{flushright}
\end{titlepage}

\tableofcontents{\tiny} %目次を自動作成するコマンド
\setcounter{page}{1} %ページ数を1とする
\pagestyle{plain}

\newpage

\section{問題}
\begin{itembox}{}
TSPをGAで解くと仮定して、コーディング/交叉法(+突然変異)を提案せよ。
\end{itembox}
\subsection{問題の要旨}
本稿では,TSP(最短経路問題)をGA(遺伝的アルゴリズム)を用いて,解く時のコーディング手法,交叉法,突然変異の方法を提案する.
\section{











\begin{thebibliography}{9} 
  \bibitem{} \url{}
\end{thebibliography}
\end{document}