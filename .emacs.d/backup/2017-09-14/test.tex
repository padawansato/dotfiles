\documentclass[9pt,fleqn]{jarticle} %記事や論文を書く際にjarticleを使用

%枠付文章とpng画像を出力するため
\usepackage{ascmac}%枠付き文字出力
\usepackage{fancybx}
\usepackage{framed}
\usepackage[dvipdfmx]{graphicx}%pdf出力のための
\usepackage{listings,jlisting}
\usepackage{amsmath,amssymb}
\usepackage{mediabb}
\usepackage{titlesec}
\titleformat*{\section}{\normalsize\bfseries}
\titleformat*{\subsection}{\normalsize\bfseries}
\usepackage{url}
\lstset{language=C,%
        basicstyle=\footnotesize,%
        commentstyle=\textit,%
        classoffset=1,%
        keywordstyle=\bfseries,%
	frame=tRBl,framesep=5pt,%
	showstringspaces=false,%
	breaklines=true,%行が長くなったときの改行。trueの場合は改行する。
        numbers=left,stepnumber=1,numberstyle=\footnotesize%
	}%
\begin{document}
\begin{titlepage}
    \begin{center} %vspaceは縦方向のスペースで,hspaceは横方向のスペース
        \fontsize{25pt}{0pt}\selectfont %フォントサイズと行送りを設定
        \vspace*{100truept}
        \bf{textest}\\ %タイトル1行目
        \vspace*{10truept}
        \bf{レポート\#2}\\ 
        \vspace{240truept}
    \end{center}
    \begin{flushright}
        {\large
            \fontsize{16pt}{0pt}\selectfont
          所  属  :琉球大学 工学部 情報工学科\\
            \vspace{5truept}
          学籍番号:155755G\\
            \vspace{5truept}
          氏  名  :佐藤 匠\\
            \vspace{5truept}
          提 出 日:\\
            \vspace{5truept}
        }    
    \end{flushright}
\end{titlepage}

\tableofcontents{\tiny} %目次を自動作成するコマンド
\setcounter{page}{1} %ページ数を1とする
\pagestyle{plain}

\newpage

%パラグラフ(段落)
%パラグラフは,1つの考えのみに留める.
%パラグラフの最初の段落は,そのパラグラフの結論(トピックセンテンス)を書く.
\section{asd}





\begin{thebibliography}{9} 
  \bibitem{} \url{}
\end{thebibliography}
\end{document}
